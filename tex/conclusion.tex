
\section{Conclusion}

In this paper we introduce \framework, a middle-end compiler for domain specific languages that separates the algorithm, the schedule, and the data layout in a four-layer intermediate representation.  \framework supports backend code generation for multicore CPUs, GPUs, FPGAs, and distributed systems, as well as machines that contain any combination of these architectures.

\framework is designed so that most DSLs can use high-level scheduling and data mapping constructs to control the lowering from the algorithm to the backend, cross-platform code.  In addition, the underlying representations are accessible to advanced users that wish to implement new optimizations and transformations.

We evaluate \framework by creating a new middle-end for the Halide and Julia compilers, targeting a variety of backends.  We also demonstrate that the transformations made possible by \framework increased performance by up to $4\times$ in Halide and $16\times$ in Julia and demonstrate that \framework{} can generate a very fast code matching one of the most hand optimized kernels (Intel MKL gemm).

% Future work includes support for more DSL compilers and adding a layer of automation on top of the current framework; for example, we want to enable automatic scheduling within \framework.
\newpage